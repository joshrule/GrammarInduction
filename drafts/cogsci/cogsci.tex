% Annual Cognitive Science Conference

\documentclass[10pt,letterpaper]{article}

\usepackage{cogsci}
\usepackage{pslatex}
\usepackage{apacite}
\usepackage{graphicx}
\usepackage{amsmath,amssymb}
\usepackage[longnamesfirst]{natbib}
\usepackage{url}

\title{Representing and Learning a Large System of Number Concepts \\ with Latent Predicate Networks}
 
\author{
  {\large \bf Joshua Rule (rule@mit.edu)}\\
  {\large \bf Eyal Dechter (edechter@mit.edu)}\\
  {\large \bf Joshua B. Tenenbaum (jbt@mit.edu)}\\
  MIT, 46-4053, 77 Massachussetts Avenue, Cambridge, MA 02139 USA}

\begin{document}

\maketitle

\begin{abstract}
  The natural numbers are one of the first abstract conceptual systems
  children acquire, forming a foundation on which the rest of
  mathematics and the sciences depend. Psychologists have accordingly
  spent decades investigating number knowledge in children and adults
  as a case study in concept representation and acquisition
  \citep{fuson1988children,galGel2005,Car2009}. Even so, psychological
  models of natural number have largely ignored two challenges related
  to the language-like productivity and compositionality of exact
  number concepts: 1) there is an unbounded set of exact number
  concepts, each with distinct semantic content; and 2) people can
  reason flexibly about any of these concepts (even fictitious ones
  like \emph{eighteen-gazillion and thirty-one}). Conventional models
  of concept learning that represent individual concepts as
  collections of prototypes or rules do not naturally explain these
  capacities ({\bf CITATIONS NEEDED}). Instead we must learn the
  structure of the entire infinite set of exact number concepts,
  focusing on how relationships between numbers support reference and
  generalization. Here, we suggest that the latent predicate network
  (LPN) -- a probabilistic context-sensitive grammar formalism --
  facilitates tractable learning and reasoning for exact number
  concepts \citep{DecRulTenming}. We show how the number words and
  their relations to one another can be expressed in this formalism
  and discuss a Bayesian learning algorithm for LPNs, suggesting a
  computational mechanism by which children might learn abstract
  numerical knowledge from linguistic utterances about numbers.

  \textbf{Keywords:}
  child development; concept learning; number; generalization;
  computational model; grammar induction;
\end{abstract}

\section{Introduction}

The natural numbers (1, 2, 3, $\ldots$) are some of the most powerful
concepts yet discovered. They allow precise quantification over finite
sets and thus form a foundation on which nearly all mathematical and
scientific intuition is built. They are among the simplest of
abstract, symbolic structures, yet their usefulness is literally
infinite.

Despite this pivotal role, current evidence suggests humans aren't
born with an innate understanding of the natural numbers
\citep{Car2009}. Number is instead laboriously acquired over the
course of early childhood, a process stretching well into grade school
\citep{Nat2010}. Even so, the natural numbers are among the first
abstract, symbolic conceptual systems we acquire. Understanding how
number is acquired - on what basis representation and by what
computational process - is far more than a simple case study and
promises to significantly increase our understanding of abstraction
and conceptual development.

Natural number acquisition has accordingly been studied intensely and
to great effect. Infant and animal studies suggest several innate
systems, while not containing explicit natural number concepts, are
important for scaffolding our initial representations of number. These
include systems for object individuation and approximate magnitude
\citep{feigenson2004core,dehaene2011number}. One of the most
well-established phenomena of number learning is that the ability to
reliably count sets of objects develops stereotypically, even among
cultures where number is traditionally unimportant
\citep{Wyn1992,JarPianSpelEtAl2014}. Initially, children are
completely unable to associate sets of a given size with the correct
number word. Then, they can do so for sets no larger than one,
followed much later by sets no larger than two, followed again by
sets no larger than three. Typically, children then appear to
generalize the procedure to the other number words they know and can
reliably count out sets of any size, provided they know a sufficiently
large count list. At this point, they are said to have acquired the
\emph{Cardinal Principle} and are variously called \emph{CP-knowers}
or \emph{full counters}.

Initial attempts to collate decades of number research into a coherent
theory or model have largely focused on the cognitive change that
helps children become \emph{full counters}
\citep{Car2009,PianGoodTen2012}. Recent work suggests, however, that
the ability to reliably count sets of objects, while closely related,
is undeniably distinct from our conceptual knowledge of numbers as
representing cardinalities of exact sets
\citep{DavEngBar2012,izard2014toward,JarPianSpelEtAl2014}. More
generally, counting a set of objects requires only a very partial
understanding of number. Both prose and computational models of
counting and number learning have also focused almost exclusively on
numbers between one and ten, presumably because it is during this
interval that the transition to \emph{full counter} occurs. Most
studies, then, have focused on the development of a specific skill
requiring limited knowledge of a small subset of numbers.

While the problems of how children link physical sets with the
counting routine and develop a concept of sets as exact collections
are crucial, we direct our attention elsewhere in this paper. We focus
on how children might acquire knowledge of an infinite number system
from language, particularly for sets which they are unlikely to ever
see counted out explicitly. We begin by discussing how to represent
the sort of conceptual knowledge needed to describe an infinite number
system, and show how a particular formalism, the Probabilistic Range
Concatenation Grammar (PRCG) can represent number this way
\citep{boullier2005range}. We then show how this grammar can be
learned using Bayesian inference in an LPN, a learning framework for
PRCGs \citep{DecRulTenming}.

\section{Representing Number Knowledge}

To show how a system of concepts like number might be learned, we must
first understand what that system of concepts is and how it might be
represented. We begin by describing several challenges a
representation of number must overcome. We then formally introduce
PRCGs as an answer to these challenges. Finally, we show how this
formalism, initially developed to explain syntactic structures in
natural language, can explain the conceptual structure of number
words.

\subsection{The Challenges of Number}

Relative to many other semantic fields children encounter ({\it e.g.}
the parts of the body, types of furniture in a house), the natural
numbers are highly distinctive.

First, whereas many other semantic fields refer to relatively concrete
classes of objects or object parts, the natural numbers refer
primarily to an abstract property (cardinality) of an abstract entity
(sets). Semantic fields like the parts of the body also tend to be
relatively limited in scope, applying primarily, in this case, to
physical parts of vertebrates. By contrast, natural number is
incredibly broad, applying not only to concrete objects, but also to
things like sets of objects ({\it e.g.}  three pairs of socks),
sounds, events, time periods, people and other agents, and numbers
themselves ({\it e.g.}  three threes makes nine). This incredibly
broad applicability means that number concepts can often be used, and
thus must be understood and represented, without direct perceptual
grounding.

Second, there are infinitely many number concepts. Even given a
practically infinite number of instantiated body parts ({\it e.g.}
Timmy's nose), the collection of names for body parts ({\it e.g.}
tail, eye, nose, tummy, ...) is decidedly finite. By contrast,
children learn far more than just the numbers one through ten. There
are not only a practically infinite number of natural number instances
({\it e.g.} three noses), but a truly infinite number of natural
number concepts ({\it e.g.} three). Moreover, being infinite, the
amount of explicitly counted, perceptually-grounded evidence children
receive relative to the size of the semantic field is incredibly
sparse (When did you last see exactly 253 objects?) In order to
accommodate such an expressive conceptual system in a finite mind, the
concepts themselves must be constructed as needed in a systematic and
compositional manner. Together with the last point, we must also
conclude that not only are numbers represented without direct
perceptual grounding, many of them must also be learned this way as
well.

Third, numbers are not uniquely used to describe cardinalities but
also have meanings related to sequencing, counting, ordinality,
measuring, and many types of non-numerical meanings ({\it e.g.}
telephone numbers) (Fuson, Richards, Briar, 1982). Even when numbers
do describe cardinalities, it could also be that interest is not so
much in the cardinality itself as in some more complex property, such
as whether it is more or less than another cardinality or how it might
operate arithmetically via addition, subtraction, multiplication or
division. Indeed, children may eventually learn about negative and
rational numbers, algebra, geometry, and myriad other mathematical
disciplines. This hugely diverse range of meanings makes it impossible
to fully describe the meaning of \emph{three} without referencing
\emph{two}, \emph{four}, and potentially all other numbers. The
concept of \emph{two} (or any other number) is not rightly understood
as a single object, but rather as a web of relationships that hold for
some unique token \emph{two}. The sum collection of these
relationships is what defines a number.

How can we hope to represent a systems of concepts which are: 1)
learnable without direct perceptual grounding; 2) compositionally
constructed; and 3) relationally defined? We propose that the
representation best suited to this sort of structure is a
grammar. Grammars can be induced directly from a stream of utterances,
are highly compositional, and define their constituents based on their
relationships to other components rather than as discrete
objects. Specifically, we propose to use Range Concatenation Grammars (RCGs),
an expressive yet tractable formalism originally developed to model
context-sensitive phenomena in natural language syntax.

\subsection{Probabilistic Range Concatenation Grammars}

RCGs describe precisely those string languages whose parse time is
polynomial in the length of the target
string~\citep{boullier2005range}. An RCG is a 5-tuple $G=(N, T, V, P,
S)$, where $N$ is a finite set of predicate symbols, $T$ is a set of
terminal symbols, $V$ is a set of variable symbols, P is a finite set
of $M \geq 0$ clauses of the form $\psi_0 \rightarrow \psi_1 \dots
\psi_M$, and $S \in N$ is the \emph{axiom}. Each $\psi_m$ is a term of
the form $A(\alpha_1, \dots, \alpha_{\mathcal{A}(A)})$, where $A \in
N$, $\mathcal{A}(A)$ is the arity of $A$, and each $\alpha_i \in (T
\cup V)^*$ is an argument of $\psi_m$. We call the left hand side term
of any clause the \emph{head} of that clause and its predicate symbol
is the \emph{head predicate}.

A string $x$ is in the language defined by an RCG if one can
\emph{derive} $S(x)$. A derivation is a sequence of rewrite steps in
which substrings of the left hand side argument string are bound to
the variables of the head of some clause, thus determining the
arguments in the clause body. If a clause has no body terms, then its
head is derived; otherwise, its head is derived if its body clauses
are derived.\footnote{This description of an RCG language technically
  only holds for \emph{non-combinatory} RCGs, in which the arguments
  of body terms contain only single variables. Since any
  \emph{combinatory} RCG can be converted into a non-combinatory RCG,
  this description suffices.}

PRCGs are RCGs where each clause $C_k \in P$ is annotated with
a probability $p_k$ such that ${\forall A \in N, \,
  \sum_{k:head(C_k)=A} p_k = 1}$. A PRCG defines a distribution over
strings $x$ by sampling from derivations of $S(x)$ according to the
product of probabilities of clauses used in that derivation.\footnote{This is a
well defined distribution as long as no probability mass is placed on
derivations of infinite length; here, we only consider PRCGs with
derivations of finite length.}

\subsection{A Grammar for Number Concepts}

\begin{figure*}[t]
  \begin{centering}
    \includegraphics[width=0.9\linewidth]{grammarOfNumber/gon.pdf}
    \caption{A Range Concatenation Grammar whose strings are valid number words.}
    \label{fig:gon}
  \end{centering}
\end{figure*}

\begin{figure*}[t]
  \begin{centering}
    \includegraphics[width=\linewidth]{parseTrees/parse.pdf}
    \caption{RCG parses for \emph{Number} (Blue), \emph{Successor} (Red), and \emph{More} (Green).}
    \label{fig:parse}
  \end{centering}
\end{figure*}

Having motivated our decision to model concepts as sets of relations
expressed by a grammar, and having described RCGs as our formalism of
choice, we now show that RCGs can in fact capture the conceptual
structure of the natural numbers.

In this initial exploration, we capture three kinds of number
knowledge. First, we show that an RCG can capture the distinction
between valid and invalid number words. While a seemingly basic task,
the understanding that twenty-nine is a number while twenty-ten is
not, is one which children struggle to learn \citep{FusRicBriar1982}.
Second, we want to show that an RCG can capture predecessor and
successor relationships. Unlike the proposed induction that children
make between final number word and the final item reached in a set
while counting, we do not attempt to link the physical and conceptual
here. Instead, we are proposing a model for an aspect of number
learning that begins earlier and continues later, that of learning the
count list itself. Third, we want to show that an RCG can go beyond
mere successor and predecessor relations to describe more complex
aspects of number such as \emph{More} and \emph{Less}. For any set of
$n$ numbers, there are exactly $n$ valid number words. Because $1$ has
no predecessor and we cannot concisely name the successor of the last
number for which we know the number word ({\it e.g.} What comes after
999,999,999 if we do not know the word ``billion''?), there are
actually only $n-1$ valid successor or predecessor relations. There
are, however, $(n^2-n)/2$ each of the more and less relations.
Mastering these more abstract concepts thus requires significantly
greater generalization.

Capturing these relations with an RCG is not only possible, but it can
be done quite compactly. Our grammar for the concepts of
\emph{Number}, \emph{Successor}, \emph{Predecessor}, \emph{Less}, and
\emph{More} covers all numbers between 0 and 1 billion, exclusive, and
requires only 216 rules. Even considering just \emph{Number},
\emph{Successor}, and \emph{Predecessor}, these 216 rules cover more
than 500 quadrillion relations. Figure \ref{fig:gon} shows a schematic
of the rules concerned with determining valid and invalid numbers,
while the rest, due to space constraints, can be found
online.\footnote{http://github.com/joshrule/GrammarInduction} This
grammar has not been optimized for compactness or efficiency. A number
of predicates could be compressed or even eliminated, for example, by
implementing \emph{More} with a binary search tree or as the
transitive closure of \emph{Successor}. Instead, we focused on
providing a grammar that would be correct, easy to understand for the
human reader, and fit a prefix-base-suffix understanding of number, as
discussed below.

Intuitively, a number word like six-hundred thirty-seven is a valid
number word because we have six units of one hundred each and
thirty-seven remaining units of one each. That is, we have some base
unit larger than one (hundred) and we track both how many of those we
have (six), and how many of the next smallest base unit (one) we have
(thirty-seven). We denote the sum of these (six-hundred +
thirty-seven) simply by concatenating the two terms from largest to
smallest base (six-hundred thirty-seven). This structure can be grown
recursively. Nine-thousand seven-hundred sixteen is created by taking
nine thousands units and tacking on the remainder, which is seven
hundreds, plus its remainder of sixteen ones: ``nine'' $\times$
``thousand'' $+$ (``seven'' $\times$ ``hundred'' + (``sixteen''
$\times$ ``one'')). Note that there is no explicit mention of the base
``one'' in the final number word - it is implied and can be marked by
appending $\varnothing$, the empty string, instead of ``one''.

Our grammar similarly uses a prefix-base-suffix description of number.
For example, to conclude that ``six-hundred thirty-seven'' is a valid
number word (Figure \ref{fig:parse}), we must show that ``six'' is a
valid prefix for ``hundred'' and ``thirty-seven'' is a valid suffix.
We must show that our use of the largest base is legal, as well as
show recursively that the rest of the number is legal. ``six'' is a
valid prefix for ``hundred'' because it is a number word representing
a \emph{ones} number, a number between one and nine. It would be
incorrect for ``hundred'' to have no prefix, and it would also be
incorrect to use a prefix larger than ``nine''. ``thirty-seven'' is a
valid suffix because it is a valid number for a previous base, in this
case $\varnothing$, the ones base. ``thirty-seven'' is one of these
numbers because it is merely the concatenation of a \emph{decade} word
and a \emph{ones} word. Thus, the compositional use of relatively
simple predicates helps us analyze the structure of a complex phrase
like ``six hundred thirty seven'' and show that while it is a valid
number word, ``hundred six seven thirty'' is not. \emph{Successor} and
\emph{More} can similarly be encoded (Figure \ref{fig:parse}), while
\emph{Predecessor} and \emph{Less} can be encoded quite simply as
$\text{Less}(X,Y) \leftarrow \text{More}(Y,X)$ and $\text{Pred}(X,Y)
\leftarrow \text{Succ}(Y,X)$.

These examples help clarify what this grammar represents and how we
accomplish that representation. First, and perhaps most importantly,
\emph{Number} contains no link between set cardinalities and number
words, and \emph{More} lacks a connection to approximate magnitude. We
intend to explore these connections in future work but focus here on
the fact that to know what the number words mean is to know how they
interact with other concepts you already have, including (potentially
innate) pre-existing conceptions of \emph{More} or \emph{Less} used to
differentiate sets, or \emph{Successor} used to memorize songs and
routines. Second, while the conceptual grammar contains a great deal
of knowledge about ``fifty-two'', there is no object that in any sense
\emph{contains the full meaning of} ``fifty-two''. The meaning is
simply the collection of relationships which hold for the token
``fifty-two''. Third, the grammar never produces nor parses anything
resembling full English sentences. This is a grammar for the structure
of concepts, not the structure of language. When attempting to parse
something like \emph{Succ(ninety nine, one hundred)}, we assume some
other system more directly involved in language preprocesses
linguistic utterances into a partially predicated state, which is then
checked against the knowledge encoded in our conceptual grammar.
Fourth, the names given to specific predicates have no inherent
semantics. \emph{Number} could just as easily be called \emph{Wug} or
\emph{Dax}. What is important is not a predicate's name, but the
relations it enters into with other predicates and thus the argument
strings for which it holds. Similarly and crucially, individual words
like ``hundred'' or ``nine'' have no inherent meaning. They acquire
meaning through the way each predicate combines or contrasts them with
other words. Finally, we provide a grammar explaining number relations
in terms of English rules for expressing cardinalities, but that
grammar could easily be modified to model different counting systems,
such as those used in French or Mandarin.

\section{Learning Number Knowledge}

%% Eyal

\subsection{Latent Predicate Networks}

%% Eyal

\subsection{Method}

%% Eyal

\subsection{Results}

%% Eyal

\section{Discussion}

%% first half: Eyal

Either way, as these models claim to cover more well-trodden
territory, they should also become correspondingly more quantitative
in their predictions. The ability to generalize as well as the
qualitative similarities we show here between LPNs and children's
overgeneralizations are intriguing, but 

%% second half of discussion

The Rational Rules model and descendent models
\citep{goodman2008rational,T.D.Ullman:2012:1b1b6,PianGoodTen2012}
share a similar vision with us of exploring concept learning through
compositional representations, Bayesian induction, and sparse
evidence. We agree that these are fundamental to understanding concept
learning. The major difference is in these models represent concepts;
both use a grammar, but they do so very differently. Rational Rules
models see concepts as specific sentences in a language defined by a
grammar. The (potentially infinite) hypothesis space over specific
concepts is a collection of logical sentences, or rules, generated by
that grammar, and the goal is to find sentences which apply to
collections of objects in the world. In our model, the hypothesis
space is not over sentences in a grammar but over possible grammars.
To wit, the grammars in this paper are only a few of millions of
possible grammars in our hypothesis space. Moreover, neither these
grammars nor sentences in this grammar represent individual concepts.
It is instead the entire language of the grammar, the network of all
possible sentences or relations which, taken as a whole, provides the
extension of certain concepts, here \emph{Number}, \emph{Successor},
\emph{Predecessor}, \emph{More}, \emph{Less}, and the other predicates
learned to support these ({\it i.e.} \emph{Decade}, \emph{Prefix},
\ldots). Where Rational Rules models see concepts as specific
sentences in a grammatical language, LPNs see concepts as networks of
grammatically-generated relations.

This comparison holds for the Rational Rules-style model of counting
and CP-acquisition given in \citep{PianGoodTen2012}. This model is
focused on finding a specific sentence which captures the CP, in this
case a program in the simply-typed lambda calculus, rather than on
finding a grammar some of whose relations correctly describe the CP.

More generally, we focus on a different psychological problem here
than in \citep{PianGoodTen2012}, namely, we are not proposing a model
of how children acquire their initial number concepts or the CP. We
have no representation of physical sets, and thus no possible link can
be made between set manipulation and a counting sequence. Since the
acquisition of 1, 2, 3, and the more general ability of \emph{full
  counters} relies on this link, our current system is not being
proposed as a model of early number acquisition. Nor are we focusing
on the related problem of discovering that sets have exact sizes.
Instead, this paper demonstrates one way that humans might learn
concepts purely from symbolic information provided by utterances.

That is not to say that our work could not be extended to provide a
more comprehensive model of number learning. We see no fundamental
incompatibility between the model presented here and extensions to
include approximate magnitude, object tracking, set manipulation, or
more complex morphology ({\it e.g.} the meaning of \emph{-illion} or
\emph{-teen}). Far from it, we see our work here as a first
demonstration of LPN's suitability for capturing a broad range of
number concepts, though whether more general models are best
approached by working strictly within this formalism or using an LPN
as one module within a more complex framework is an open question.

Certainly, the human mind is more powerful than an RCG and is at least
Turing-complete. RCGs provide a tractable way, however, to explore a
restricted subclass of problems. The strategies and solutions we
discover here are also available in Turing-complete systems, and in
fact implemented in one (PRISM Prolog) so our findings easily
generalize to more expressive grammars.

More broadly, we see this paper as growing out of the hypothesis that
much of human learning, including the explosion of knowledge that
occurs during development, can be explained as induction in a formal
language. This vision of the \emph{child-as-hacker} draws on and
extends the notion of the \emph{child-as-scientist}; not only are
children forming theories about the world, but they are simultaneously
developing the very conceptual language they use to formulate those
theories.

% Boullier 2003 shows that RCGs can compute things as sophisticated
% as prime number detection, though by very different means than we
% use here. Even so, understanding how to bring these abilities into a
% symbolic or mixed symbolic/iconic system is an open question.
% similar to Boullier, 2003 in that the interest is in using RCGs to
% capture mathematical concepts. The approach taken here is very
% different though. We're using a fully symbolic, rather than an
% iconic, representation of number. Boullier's work may be
% informative, though for understanding approximate magnitudes and
% other iconic forms of representation.

\section{Acknowledgments}

The authors benefited significantly from conversations with Timothy
O'Donnell and Leon Bergen. This material is based upon work supported
by the Center for Minds, Brains and Machines (CBMM), funded by NSF STC
award CCF-1231216, an NSF Graduate Research Fellowship, and the Eugene
Stark Graduate Fellowship.


\bibliographystyle{apacite}

\setlength{\bibleftmargin}{.125in}
\setlength{\bibindent}{-\bibleftmargin}
\bibliography{cogsci}

\end{document}
