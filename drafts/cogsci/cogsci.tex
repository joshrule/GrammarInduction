% Annual Cognitive Science Conference

\documentclass[10pt,letterpaper]{article}

\usepackage{cogsci}
\usepackage{pslatex}
\usepackage{apacite}
\usepackage{graphicx}
\usepackage{amsmath,amssymb}
\usepackage{natbib}
\usepackage{url}

\title{Representing and Learning a Large System of Number Concepts using Latent Predicate Networks}
 
\author{{\large \bf Joshua Rule, Eyal Dechter, Joshua B. Tenenbaum} \\
  $\{$rule, edechter, jbt$\}$ @ mit.edu \\
  Department of Brain and Cognitive Sciences\\
  77 Massachussetts Avenue, Cambridge, MA 02139 USA}

\begin{document}

\maketitle

\begin{abstract}
  The natural numbers are one the first abstract conceptual systems
  children acquire, forming a foundation on which the rest of
  mathematics and the sciences depend. Psychologists have accordingly
  spent decades investigating number knowledge in children and adults
  as a case study in concept representation and acquisition
  \citep[{\it e.g.}][]{Car2009}. To the extent that formal
  psychological models of natural number have been studied, however,
  they have largely ignored at least two challenges related to the
  language-like productivity and compositionality of exact number
  concepts: 1) there is an unbounded set of exact number concepts,
  each with distinct semantic content; and 2) people can reason
  flexibly about any of these concepts (even fictitious ones like
  \emph{eighteen-gazillion and thirty-one}). Conventional models of
  concept learning that represent individual concepts as collections
  of prototypes or rules do not naturally explain these capacities
  (CITATION NEEDED). Instead we need a way of learning the structure
  of the entire infinite set of exact number concepts as a system akin
  to a natural language grammar. The heart of what must be learned are
  the relationships between concepts that support reference and
  generalization. Here, we suggest that the latent predicate network
  (LPN) -- a probabilistic context-sensitive grammar formalism we have
  previously developed -- facilitates tractable learning and reasoning
  for exact number concepts \citep{DecRulTenming}. We show how the
  grammar of the number words and their relations to one another can
  be expressed in this formalism and discuss a Bayesian learning
  algorithm for LPNs, suggesting a computational mechanism by which
  children might learn abstract numerical knowledge from linguistic
  utterances about numbers.

  \textbf{Keywords:}
  child development; concept learning; number; generalization;
  computational model; range concatenation grammar;
\end{abstract}

\section{Introduction}

The natural numbers (1, 2, 3, $\ldots$) are some of the most powerful
concepts that humans have discovered. They allow precise
quantification over finite sets and thus form a foundation on which
nearly all mathematical and scientific intuition is built. They are
among the simplest of abstract, symbolic structures, yet their
usefulness is literally infinite.

Despite this pivotal role, current evidence suggests that humans
aren't born with any innate understanding of the natural numbers
\citep{Car2009}. Instead, number is laboriously acquired over the
course of early childhood, a process stretching well into grade school
\citep{Nat2010}. Even so, our earliest understanding of the natural
numbers is among the first abstract, symbolic conceptual systems we
acquire. Understanding how number is acquired - on what basis
representation and by what computational process - thus promises to
increase our understanding of abstraction and conceptual development
in a much deeper way than a typical case study.

Natural number acquisition has accordingly been studied intensely for
the last several decades, to great effect. Strong evidence suggests
several innate systems, while not containing explicit natural number
concepts, are important for scaffolding our initial representations of
number. These include systems for parallel object individuation and
approximate magnitude \citep{feigenson2004core}. One of the most
well-established phenomena of number learning is that the ability to
reliably count sets of objects develops in a highly stereotypical
process, even among cultures where number is traditionally unimportant
\citep{Wyn1992,JarPianSpelEtAl2014}. Initially, children are
completely unable to associate sets of a given size with the
appropriate number word. Then, they can do so for sets no larger than
one, followed some time later by sets no larger than two, followed
again by sets no larger than three. Typically, children then appear to
generalize the procedure to the other number words they know and can
reliably count out sets of any size, provided they know a sufficiently
large count list. At this point, they are said to have acquired the
\emph{Cardinal Principle} and are variously called \emph{CP-knowers}
or \emph{full counters}.

Initial attempts to collate decades of number research into a coherent
theory or model have largely focused on the cognitive change that
helps children become \emph{full counters}
\citep{Car2009,PianGoodTen2012}. Recent work suggests, however, that
the ability to reliably count sets of objects, while closely related,
is undeniably distinct from our conceptual knowledge of numbers as
representing cardinalities of exact sets
\citep{DavEngBar2012,izard2014toward,JarPianSpelEtAl2014}. More
generally, the ability to count a set of objects requires only a very
partial understanding of the natural numbers. Both prose and
computational models of counting and number learning have also focused
almost exclusively on numbers between one and ten, presumably because
it is during that interval that the transition to \emph{full counter}
occurs. Most studies, then, have focused on the development of a
specific skill requiring limited number knowledge for a small subset
of numbers.

Children acquire far more than just the numbers one through ten; they
eventually learn that the numbers are infinite and can be considered
as compositionally constructed from a small set of more basic number
concepts according to a base system, typically base ten. Moreover,
they learn many of these number concepts from linguistic utterances
alone, without ever seeing them as explicitly counted,
perceptually-grounded sets (When did you last see exactly 253
objects?). They also learn far more than just the names of the numbers
and their ordering on the number line. They learn significantly more
complex operations like \emph{More Than} and \emph{Less Than}, as well
as simple arithmetic in the form of addition, subtraction,
multiplication and division. Indeed, they may eventually learn about
negative and rational numbers, modular arithmetic, algebra, geometry,
and myriad other mathematical disciplines.

While the problems of how children link perceptually grounded sets
with the counting routine and develop a concept of sets as exact
collections are crucial, we direct our attention elsewhere in this
paper. Specifically, we focus on how children might be able to acquire
number knowledge from language, particularly for sets which they are
unlikely to ever see counted out explicitly. We begin by discussing
how to represent the sort of conceptual knowledge needed to describe
an infinite number system, and showing how a particular formalism,
\citeauthor{boullier2005range}'s (\citeyear{boullier2005range})
Probabilistic Range Concatenation Grammar (PRCG) can be used to
represent number knowledge this way. We then show how portions of this
grammar can be learned using Bayesian inference in a Latent Predicate
Network (LPN), a learning framework for PRCGs which we have previously
developed \citep{DecRulTenming}.

\section{Representing Number Knowledge}

To show how a system of concepts like number might be learned, we must
first understand what that system of concepts is and how it might be
represented. We begin by describing several challenges a
representation of number must overcome. We then formally introduce
Probabilistic Range Concatenation Grammars as an answer to these
challenges. Finally, we show how this formalism, initially developed
to explain syntactic structures in natural language, can explain the
conceptual structure of number words.

\subsection{The Challenges of Number}

Relative to many other semantic fields a child encounters early in
life ({\it e.g.} the parts of the body, types of furniture in a
house), the natural numbers are highly distinctive.

First, whereas many other semantic fields refer to particular,
relatively concrete classes of objects or object parts, the natural
numbers refer to an abstract property (cardinality) of an abstract
entity (sets). Where semantic fields like the parts of the body are
relatively limited in scope, applying primarily, in this case, to
physical parts of vertebrates. By contrast, natural number is
incredibly broad, applying not only to concrete objects, but also to,
among other things, sets of objects ({\it i.e.}  three pairs of
socks), sounds, events, time periods, people and other agents, and
numbers themselves ({\it i.e.}  three threes makes nine). This
incredibly broad applicability means that number concepts can often be
used, and thus must be understood and represented, without direct
perceptual grounding.

Second, there are infinitely many number concepts. Even though there
is a practically infinite number of instantiated body parts ({\it
  i.e.} Timmy's nose), the collection of names for body parts ({\it
  i.e.} tail, eye, nose, toe, arm, tummy, ...) is decidedly finite. By
contrast, there are not only a practically infinite number of natural
number instances ({\it i.e.}  three apples), but a truly infinite
number of natural number concepts. In order to accommodate such an
expressive conceptual system in a finite mind, the concepts themselves
must be constructed as needed in a systematic and compositional
manner.

Third, numbers are not uniquely used to describe cardinalities but
also have meanings related to sequencing, counting, ordinality, and
measuring, as well as many types of non-numerical meanings ({\it e.g.}
in telephone numbers) (Fuson, Richards, Briar, 1982). Even when
numbers do describe cardinalities, those cardinalities may not be
determined by simple counting, but by removing items from a known set,
combining multiple sets into a single set, or any number of other
ooperations. This hugely diverse range of meanings makes it impossible
to fully describe the meaning of ``three'' without reference to
``two'', ``four'', and potentially all other numbers. The concept of
``two'' (or any other number), we contend, is not rightly understood
as a single object, but rather as a web of relationships that hold for
some unique token ``two''. The sum collection of these relationships,
which include \emph{More Than}, \emph{Half Of}, \emph{Immediately Preceding},
and \emph{Is Prime}, is what defines a number.

How can we hope to represent a systems of concepts which are: 1)
learnable without direct perceptual grounding; 2) compositionally
constructed; and 3) relationally-defined? We propose that the
representation best suited to explaining this sort of structure is a
grammar. Grammars can be induced directly from a stream of utterances,
are highly compositional, and define their constituents ({\it e.g.}
noun phrases) based on their relationships to other components rather
than as discrete objects. Specifically, we propose to use Range
Concatenation Grammars, an expressive yet tractable formalism
originally developed to explain context-sensitive phenomena in natural
language syntax.

\subsection{Probabilistic Range Concatenation Grammars}

Range Concatenation Grammars (RCGs) describe precisely those string
languages whose parse time is polynomial in the length of the target
string~\citep{boullier2005range}. An RCG is a 5-tuple $G=(N, T, V, P, S)$, where $N$ is a finite set of predicate symbols, $T$ is a set
of terminal symbols, $V$ is a set of variable symbols, P is a finite
set of $M \geq 0$ clauses of the form $\psi_0 \rightarrow \psi_1 \dots
\psi_M$, and $S \in N$ is the \emph{axiom}. Each $\psi_m$ is a term of
the form $A(\alpha_1, \dots, \alpha_{\mathcal{A}(A)})$, where $A \in
N$, $\mathcal{A}(A)$ is the arity of $A$, and each $\alpha_i \in (T
\cup V)^*$ is an argument of $\psi_m$. We call the left hand side term
of any clause the \emph{head} of that clause and its predicate symbol
is the \emph{head predicate}.

A string $x$ is in the language defined by an RCG if one can
\emph{derive} $S(x)$. A derivation is a sequence of rewrite steps in
which substrings of the left hand side argument string are bound to
the variables of the head of some clause, thus determining the
arguments in the clause body. If a clause has no body terms, then its
head is derived; otherwise, its head is derived if its body clauses
are derived.\footnote{This description of the language of an RCG
  technically only holds for \emph{non-combinatory} RCGs, in which the
  arguments of body terms can only contain single variables. Since any
  \emph{combinatory} RCG can be converted into a non-combinatory RCG
  and we only consider non-combinatory RCGs here, this description
  suffices.}

PRCGs are RCGs where each clause $C_k \in P$ is annotated with
a probability $p_k$ such that ${\forall A \in N, \,
  \sum_{k:head(C_k)=A} p_k = 1}$. A PRCG defines a distribution over
strings $x$ by sampling from derivations of $S(x)$ according to the
product of probabilities of clauses used in that derivation.\footnote{This is a
well defined distribution as long as no probability mass is placed on
derivations of infinite length; here, we only consider PRCGs with
derivations of finite length.}

\subsection{A Grammar for Number Concepts}

\begin{figure*}[t]
		\includegraphics[width=\linewidth]{grammarOfNumber/gon.pdf}
		\caption{A Range Concatenation Grammar whose strings are valid number words.}
		\label{fig:gon}
\end{figure*}


Having motivated our decision to model concepts with a formalism
originally designed to describe natural language syntax, and having
described RCGs as our formalism of choice, we now want to show that
RCGs can in fact capture conceptual knowledge of the natural numbers.

In this initial exploration, we want to capture three kinds of number
knowledge. First, we want to show that an RCG can capture the
distinction between valid and invalid number words. While it may seem
extraordinarily basic, the understanding that fifty-nine is a number
while fifty-ten is not, is one which children struggle to learn
\citep{FusRicBriar1982}.  Second, we want to show that an RCG can
capture predecessor and successor relationships. Unlike the proposed
induction that children make between successive count and the addition
of an item into a set, we do not attempt to link the physical and
conceptual here. Instead, we are proposing a model for an earlier
aspect of number learning, that of learning the count list
itself. Third, we want to show that an RCG can go beyond mere
successor and predecessor relations to describe more complex aspects
of number such as \emph{More Than} and \emph{Less Than}. These sorts
of concepts require significantly greater generalization and thus are
signficantly more abstract, as demonstrated by the following
argument. For any set of $n$ numbers, there are exactly $n$ valid
number words. Because $1$ has no predeccessor and we cannot concisely
name the successor of the last number for which we know the number
word ({\it e.g.}  What comes after 999,999,999 if we do not know the
word ``billion''?), there are actually only $n-1$ valid successor or
predecessor relations. There are, however, $(n^2-n)/2$ each of
the more and less relations, respectively.

Capturing these relations with an RCG is not only possible, but it can
be done quite compactly. Our grammar for the concepts of \emph{Is A
  Number}, \emph{Successor}, \emph{Predeccessor}, \emph{Less Than},
and \emph{More Than} covers all numbers between 0 and 1 billion,
exclusive, and requires only XXX rules. Figure \ref{fig:gon} shows a
schematic of the rules concerned with determining valid and invalid
numbers, while the rest, due to space constraints, can be found
online.\footnote{http://github.com/joshrule/GrammarInduction}

\begin{figure*}[t]
		\includegraphics[width=\linewidth]{parseTrees/parseNumber.pdf}
		\caption{A parse for the fact that ``six-hundred thirty-seven'' is a valid number word.}
		\label{fig:parseNumber}
\end{figure*}

\begin{figure*}[t]
		\includegraphics[width=\linewidth]{parseTrees/parseSucc.pdf}
		\caption{A parse for the fact that ``one-hundred'' immediately succeeds ``ninety-nine''.}
		\label{fig:parseSucc}
\end{figure*}

\begin{figure*}[t]
		\includegraphics[width=\linewidth]{parseTrees/parseMore.pdf}
		\caption{A parse for the fact that ``one-hundred one'' is more than ``fifty-two''.}
		\label{fig:parseMore}
\end{figure*}

Figure \ref{fig:parseNumber} shows how those rules can be applied to
explain why six-hundred thirty-seven is a valid number word using the
base word ``hundred''. Our grammar makes use of a prefix-base-suffix
understanding of number, where any number can be understood as some
prefix multiplied by a base and added to a suffix; ``six-hundred
thirty-seven'' is ``six'' $\times$ ``hundred'' $+$ ``thirty
seven''. We prove that ``six-hundred thirty-seven'' is a valid number
word by showing that ``six'' is a valid prefix for ``hundred'' and
that ``thirty-seven'' is a valid suffix. We in turn show that ``six''
is a valid prefix for ``hundred'' by showing that it is a number word
representing a \emph{ones} number, that is, a number between one and
nine. It would be incorrect for ``hundred'' to have no prefix, and it
would also be incorrect to use a prefix larger than nine. To show that
``thirty-seven'' is a valid suffix, we show that ``thirty-seven'' is a
valid number for the previous base, in this case, the empty base,
$\varnothing$. These are the numbers between one and ninety-nine. We
prove that ``thirty-seven'' is one of these numbers by showing that it
is merely the concatenation of a \emph{tens} word and a \emph{ones}
word. Thus, the compositional use of relatively simple predicates
helps us analyze the structure of a complex phrase like ``six hundred
thirty seven'' and show that while it is a valid number word,
``hundred six seven thirty'' is not. Figures \ref{fig:parseSucc} and
\ref{fig:parseMore} show how \emph{Successor} and \emph{More Than},
respectively, can similarly be encoded. Note that \emph{Predecessor}
and \emph{Less Than} can be encoded quite simply as $\text{Less}(X,Y)
\leftarrow \text{More}(Y,X).$ and $\text{Pred}(X,Y) \leftarrow
\text{Succ}(Y,X).$

These figures also provide a good place to clarify exactly what we
propose for this grammar to represent and how we accomplish that
representation. First, while the conceptual grammar allows us to show
that certain relations hold for the token ``fifty-two'', there is no
object that in any sense \emph{contains the full meaning of}
``fifty-two''. The meaning is simply the sum total of the
relationships which hold for the token ``fifty-two''. Second, the
grammar never produces nor parses anything resembling full English
sentences. This is a grammar for the structure of concepts, not the
structure of language. When attempting to parse something like
``Succ(ninety nine, one hundred)'', we are assuming that some other
system, more directly involved in language processing, preprocesses
linguistic utterances into a partially predicated state, which is then
checked against the current state of knowledge represented by the
conceptual grammar. Third, the names given to specific predicates in
these figures have no semantic meaning on their own. ``Number'' could
just as easily be called ``Snickerdoodle'' or ``Dax''. What is
important is not the name of a predicate, but the relations it enters
into with other predicates and, as a result, the argument strings for
which it holds. Similarly and crucially, individual words like
``hundred'' or ``nine'' have no inherent meaning. They acquire their
meaning because of the way each predicate combines or contrasts them
with other words.

\section{Learning Number Knowledge}

the challenges

\subsection{Latent Predicate Networks}

\subsection{Method}

\subsection{Results}

\section{Discussion}

Caveats: not optimized for compactness or efficiency, but for human
readability and the intuitions of the authors, doesn't model
acquisition of initial number concepts in children (but see the
discussion for more on that front), mind is more powerful than an RCG,

by no means optimal

more sophisticated predicates

extending to approximate magnitude, object tracking, and perceptual grounding

\subsection{Related Work}

\section{Acknowledgments}

The authors benefited significantly from conversations with Timothy
O'Donnell and Leon Bergen. This material is based upon work supported
by the Center for Minds, Brains and Machines (CBMM), funded by NSF STC
award CCF-1231216. Funding for this work was also received from an NSF
Graduate Research Fellowship, and the Eugene Stark Graduate
Fellowship.


\bibliographystyle{apacite}

\setlength{\bibleftmargin}{.125in}
\setlength{\bibindent}{-\bibleftmargin}
\bibliography{cogsci}

\end{document}
